\author[Lecturer1]{Brendon J. Brewer\\
Department of Statistics, The University of Auckland}

\chapter{Bayesian Inference How-To}

\section{Introduction}


Any particular application of Bayesian inference involves making choices
about what data you are analysing, what questions you are
asking, and what assumptions you are willing to make. Once you have made all
of these choices, you are then faced with the question of how to calculate the
results. Often, this involves numerical methods. The most powerful and
popular numerical techniques are the Markov Chain Monte Carlo methods, often
abbreviated as MCMC.

There is a huge variety of Markov Chain Monte Carlo methods, and it would be
unwise to try to cover them all in this winter school. Therefore I will focus
on a small number of methods that are simple to implement, yet quite powerful
and widely applicable.





%Bla bla bla \citep{lecturer1:abreu10}, \citep{lecturer1:abreu100}

%\input lecturer1/lecturer1.bbl

